\section{Conclusion}
	For the future it can be generally assumed that complexity of software
	increases further. Especially the paradigm of parallel programming becomes
	more and more important (for example in the relatively new programming
	language \emph{Go} which offers \enquote{explicit support for concurrent
	programming.} \cite{golang}). In this work a permission model was added to
	the already established analysing technique of representing pointer
	structures as hypergraphs. This allows analysing programs with parallel
	execution and therefore a programming language was introduced which supports
	simple pointer manipulation as well as parallel execution by fork and join
	statements. The semantics were appropiatly defined in terms of hypergraph
	transformations. As long as valid contracts for the different programs that
	can be forked are provided the data race freedom for the analysed states is
	proven. But the defined permission model is actually meant as basis for a
	larger framework which allows computation of those contracts in order to
	avoid defining those by hand which is error-prone.  Thus the presented
	analysis is meant for providing a basis that allows further work with the
	framework (as presented in Section \ref{sec:futurework}). Other
	contributions regard the applied abstraction by \aclp{HRG}, namely
	computation of reachability and especially the compatibility of \acl{HR}
	with the permission model as shown in the proof of overapproximation of the
	transition relation.
