\section{Data Race Freedom}
	The main result of incorporating permissions into the heap representation
	is to avoid data races. This section proves the absence of dataraces for
	valid executions, i.e. executions that does not yield a $\bot$ symbol.
	To improve the readability some considerations are given up front:
	\begin{enumerate}
		\item Edges are only altered if the permission of this edge is
			$\mathit{WR}$ as it can be easily seen from the transition rules on
			page \pageref{transitionrules}.
		\item Selectors can only be read if the corresponding edge is present in
			the representation of the heap, which follows likewise from the
			transition rules on page \pageref{transitionrules}
		\item Let $\Tok(H) = \{ T \mid \text{T is a token and either }\exists
				e\in E_{H}.\perm_{H}(e)=\rho-\Phi \land T\in\Phi
			\text{ or }\exists e'\in E_{H}.\lab_{H}(e')=N_{T}\}$ denote the set
			of all tokens in $H$. The set of tokens is called \emph{consistent}
			if they are all disjoint and all process identifier in a token refer
			to the same process.
		\item $\mathit{BasePerms}$ that are once starred can never be
			\enquote{un-starred} again which satisfies to drop accounting for
			permissions with starred $\mathit{BasePerms}$ (as it can be seen
			in the property \textbf{Derived Tickets} only not starred permissions
			are examined).
	\end{enumerate}
	\subsection{Proof Obligation}
	It is shown by induction of the transition relation that for every
	\emph{\ac{HC}} $H$ of the transition system described by $\rhd$ holds:
	\begin{description}
		\item[Consistency of Tokens] that $\Tok(H)$ is consistent
		\item[Uniqueness of Access] that there is only one process for which the
			$\mathit{BasePerm}$ of an edge in the heap representation is
			$\mathit{WR}$ or $\mathit{WR}^{\ast}$, i. e. write access is always
			transferred completely
		\item[Derived Tickets] that all derived tickets for read access for edges
			with a $\mathit{BasePerm}$ that is either $\mathit{WR}$ or
			$\mathit{RD}$ are properly accounted in the $\mathit{PermSet}$, where
			properly accounted means that for every derived ticket the token of
			the process is given in the $\mathit{PermSet}$
	\end{description}
	For this induction only the fork and join statement as well as the
	allocation and assignment of process identifier is examined, because all
	other statements cannot alter any permission or placeholder.

	Let as induction hypothesis for all following proofs $H$ be a heap
	representation which satisfies \textbf{Consistency of Tokens} and
	\textbf{Uniqueness of Access} and \textbf{Derived Tickets}.
	\begin{proof}[Allocation]
		For $(\text{new}(x), H)\rhd(\varepsilon,
		\underbrace{H[+v][x\hookrightarrow v]}_{H'})$ it holds that there
		are new edges for used selector attached to the node $v$. All these edges
		cannot accessed by any other process other than
		the current one. Therefore $H'$ satisfies \textbf{Uniqueness of Access}.
		Furthermore $H'$ satisfies \textbf{Consistency of Tokens} by induction
		hypothesis and because no new tokens are added. Since for the new
		selectors no access tickets are yet derived the empty $\mathit{PermSet}$
		is a proper accounting, yielding \textbf{Derived Tickets}.
	\end{proof}

	\begin{proof}[Assignment of Process Identifier]
		For $(t = t', H)\rhd(\varepsilon, H[t = t'])$ it holds that $H[t = t']$
		satisfies \textbf{Consistency of Tokens} because $[t = t']$ implicitly
		contains $[\downarrow t]$ which ensures, that $t$ is removed from all
		tokens and afterwards it is added to the token that refers to the process
		$t'$ refers to (by induction hypothesis there is one unique token
		$T_{t'}$ to which $t$ is added).

		Additionally, \textbf{Uniqueness of Access} follows immediatly from the
		induction hypothesis since it is only possible that $\mathit{BasePerms}$
		are starred but no access tickets are somehow transferred between
		processes.
		
		And finally, \textbf{Derived Tickets} holds because analogous to the
		argumentation of \textbf{Consistency of Tokens} the $\mathit{BasePerm}$
		of all permission for which no proper accounting can be guaranteed, i.e.
		those for which the access ticket was represented by $\{t\}$, are
		starred (since this derived ticket can never be regained). For the others
		the proper accounting follows inherently from the induction hypothesis
		and because the tokens referring to the different processes are
		consistent (\textbf{Consistency of Tokens}).
	\end{proof}

	\begin{proof}[Fork]
		For\\[-\baselineskip] $(t = fork(m(x_{1},\dots,x_{n})), H)\rhd
		(\varepsilon, \overbrace{H[\downarrow t][\setminus E_{C}]
		[+_{\mathit{WR}}N_{\{t\}}\rightrightarrows
		\enum_{b_{C}(H)}][(E_{P_{C}}\setminus E_{C})-\{t\}]}^{H'})$
		let $C = (P_{C}, E_{C}, Q_{C})\in\Cont(m)$ be the contract by which the
		new process is forked.
		
		\textbf{Consistency of Tokens} is ensured for
		$H'$ because again $[\downarrow t]$ ensures that $t$ is removed from all
		used tokens and therefore $\{t\}$ can be added safely by the introduction
		of the placeholder $N_{\{t\}}$ and $[(E_{P_{C}}\setminus E_{C})-\{t\}]$.
		Note further that $\{t\}$ is a consistent token since $t$ is the only
		process identifier referring to the newly forked process. Futhermore by
		the definition of the initial heap state it satisfies \textbf{Consistency
		of Tokens} as well since all its permissions are simple at the
		beginning which implies an empty set of tokens.

		\textbf{Uniqueness of Access} can be assured because by the
		definition of how the initial heap state is obtained it follows that only
		for edges from $E_{C}$ the permission is $\mathit{WR}$ (for all others
		the permission is $\mathit{RD}$). Since
		$E_{C}\subseteq E^{P_{C}}_{\mathit{WR}}$ by the
		premise for contracts and because $H$ satisfies
		\textbf{Uniqueness of Access} the transformation $[\setminus E_{C}]$
		ensures \textbf{Uniqueness of Access} for $H'$ because the edges for
		which the initial heap state has $\mathit{WR}$ access are removed for
		$H'$. Furthermore this argument also yields \textbf{Uniqueness of Access}
		for the initial heap state of the forked process as well.

		It can also be seen that $H'$ satisfies \textbf{Derived Tickets} since
		for all $e\in E_{H[\downarrow t]}$ with a $\mathit{BasePerm}$ that is
		either $\mathit{WR}$ or $\mathit{RD}$ \textbf{Consistency of Tokens}
		ensures a proper accounting of the derived access tickets. And
		all reachable edges are either removed $[\setminus E_{C}]$ or the token
		$\{t\}$ is added which accounts the the newly derived access ticket for
		all reachable edges that are not removed.
		Concludingly, $H'$ satisfies \textbf{Derived Tickets}. Also for the
		initial heap state all $\mathit{PermSets}$ are empty and because there
		is not yet any derived ticket for the initial heap state it also
		satisfies \textbf{Derived Tickets}.
	\end{proof}

	\begin{proof}[Join]
		For $(\text{join}(t), H)\rhd(\varepsilon,H')$
		let $C = (P_{C}, E_{C}, Q_{C})$ denote the contract by which the joined
		process was forked intially. Furthermore it holds that\\ $Q_{1} = Q_{C}
		[\downarrow\mathit{enum}_{\mathit{Var}'_{\text{thread}}}(1)]\dots
		[\downarrow\mathit{Var}'_{\text{thread}}
		(|\mathit{Var}'_{\text{thread}}|)]$ and $Q_{2} = Q'_{1}[\setminus
		(E^{Q'_{1}}_{\mathit{RD}}\cup E^{Q'_{1}}_{\mathit{RD^{\ast}}})]$.
		It is shown for the fork statement that the initial heap state of
		forked processes satisfies \textbf{Consistency of Tokens},
		\textbf{Uniqueness of Access} and \textbf{Derived Tickets}.
		It is stated that for contracts $Q_{C}$ is obtained by a valid
		execution from the initial heap state. This valid execution implies
		preservation of \textbf{Consistency of Tokens},
		\textbf{Uniqueness of Access} and \textbf{Derived Tickets},
		thus $Q_{C}$ satisfies these properties. Furthermore after having dropped
		all process identifier ($[\downarrow t']$ for all $t'\in \PI'$) it can be
		assured that $\Tok(Q_{1}) = \emptyset$ which is therefore consistent.
		Additionally since from $Q_{1}$ to $Q_{2}$ only edges are removed it
		follows that $\Tok(Q_{2}) = \emptyset$ which is also consistent.
		Also \textbf{Uniqueness of Access}
		is preserved for $Q_{1}$ and $Q_{2}$ because all $\mathit{BasePerms}$ can
		only alter from $\mathit{WR}$ to $\mathit{WR}^{\ast}$ or from
		$\mathit{RD}$ to $\mathit{RD}^{\ast}$ for every $[\downarrow t]$
		transformation. Therefore because $Q_{C}$
		satisfies \textbf{Uniqueness of Access} so do $Q_{1}$ and especially
		$Q_{2}$. Also, because those edges that preserve an unstarred
		$\mathit{BasePerm}$ through the dropping of all process identifier must
		have had an empty $\mathit{PermSet}$ before and therefore the permission
		did not change from $Q_{C}$. Therefore $Q_{1}$ inherently satisfies
		\textbf{Derived Tickets} and because $Q_{2}$ is a subgraph of $Q_{1}$ it
		does too.

		Distinguish two cases in the following: firstly
		$E^{Q_{1}}_{\mathit{RD}^{\ast}}\neq\emptyset$, which implies that
		$H[\downarrow T_{t}]\xRightarrow{N_{T_{t}}\rightarrow Q_{2}}H'$. Then
		$H'$ satisfies \textbf{Consistency of Tokens} because the token $T_{t}$
		is removed from the token set (by successively dropping $t'\in T_{t}$ and
		replacing the placeholder $N_{T_{t}}$ by $Q_{2}$) and the corresponding
		process terminated.  Furthermore, integrating $Q_{2}$ does not add any
		token (recall that $\Tok(Q_{2}) = \emptyset$).

		\textbf{Uniqueness of Access} can be assured because $Q_{C}$ satisifes
		\textbf{Uniqueness of Access} and thus, integrating those edges with
		$\mathit{BasePerm}$ $\mathit{WR}$ or $\mathit{WR}^{\ast}$ into $H$ hands
		over the write ticket to $H$. The write ticket was unique before and
		since the process that hands the write ticket over terminated it is
		unique afterwards.

		For \textbf{Derived Tickets} it is noteworthy that it cannot be
		guaranteed that all derived access tickets are completely returned since
		there is at least one edge $e\in E^{Q_{1}}_{\mathit{RD^{\ast}}}$ (thus
		this edge can potentially be accessed concurrently by a process and this
		process can never be joined since the refernce to it is lost). Because
		the initial heap state is only provided with $\mathit{WR}$ or
		$\mathit{RD}$ permission this implies a loss of information regarding the
		accounting in the forked process.
		This loss of information propagates strictly by dropping the token that
		identifies the derived access ticket. This ensures a proper accounting
		since those permissions from which this access ticket was derived are
		starred (more precisely their $\mathit{BasePerm}$). Furthermore for the
		\enquote{WR-part} \textbf{Derived Tickets} is shown above which concludes
		that $H'$ satisfies \textbf{Derived Tickets}.

		Secondly, examine the case that
		$E^{Q_{1}}_{\mathit{RD}^{\ast}}=\emptyset$ such that
		$H[\downarrow T_{t}]\xRightarrow{N_{T_{t}}\rightarrow Q_{2}}H'$:
		it follows analogous to the first case that $H'$ satisfies
		\textbf{Consistency of Tokens} and \textbf{Uniqueness of Access}. For
		the \textbf{Derived Tickets} it is important that the whole access ticket
		is returned because since $Q_{C}$ satisfies \textbf{Derived Tickets} the
		empty $\mathit{PermSet}$ for the permissions with $\mathit{RD}$ as
		$\mathit{BasePerm}$ guarantees that in the execution every granted access
		ticket is completely recollected (in other words there is no process
		directly or indirectly\footnote{since in the given setup process
		can only be joined by the process that forked them, there is a distinct
		predecessor relationship between processes. Indirectly forked refers to
		any process which is in the transitive closure of the predecessor
		relationship to the examined process}
		forked from the joined process that can access these edges). 
		This means the granted access ticket (represented as
		$\mathit{RD}$ permissions in the initial heap state) are completely
		returned and it follows, because the \enquote{WR-part} satisfies
		\textbf{Derived Tickets} that $H'$ satisfies \textbf{Derived Tickets}.
	\end{proof}

	Concludingly, this induction yields data race freedom by the following
	argument: Because proper accounting for all edges with a permission that has
	a $\mathit{BasePerm}$ of $\mathit{WR}$ or $\mathit{RD}$ is guaranteed
	(\textbf{Derived Tickets}) it follows that any derived access ticket from
	this permission restricts to read access (by the definition of the
	transition rules). A permission of $\mathit{WR}$ guarantees exclusive access
	because it meets the condition for proper accounting of \textbf{Derived
	Tickets}, thus its empty $\mathit{PermSet}$ ensures that no other process
	an access this value concurrently.  $\blacksquare$
