\section{Programming Language}
	\label{sec:ProgLang}
	In the following a very basic programming language is discussed, which
	allows simple \emph{heap} manipulating operations. It is similar to the
	programming language introduced in \cite{fmsd}. For the manipulation of the
	heap it is possible to create objects by allocation. All these objects
	contain a set of \emph{selectors} which refer to other objects or a
	distinguishable empty state, \textbf{null}. Especially
	there is just one set of selectors ($\Sel$) and all objects are universally
	typed in the sense that the selectors of all objects are $\Sel$.
	Additionally every program has a set of variables ($\Var$) which refer to
	objects within the heap. This heap is implicitly \emph{garbage collected},
	i.e. objects that can not be reached from one of the variables are
	deleted. Initially all variables and selectors of newly allocated objects
	refer to the \textbf{null} value.
	One of the main features of the presented programming
	language is the possibility of parallel execution. Therefore the \emph{fork}
	and the \emph{join} statement are introduced. On the one hand the fork
	statement allows to spawn a new process and on the other hand the join
	statement synchronises a process by blocking until the process that is
	joined terminates.
	Every process operates on variables which are exclusively accessible to
	itself.  But the objects of the heap can be shared between different
	processes therefore at the forking of a new process it is provided with a
	set of parameters which refer to objects of the heap. These parameters are
	the initial values for some variables of the newly forked process. Because a
	program might start multiple processes there is a set of \emph{process
	identifier} ($\PI$) which are used to identify forked processes. These
	process identifier are also unique for every executing process (as variables
	are) and the join statement is called with one of theses identifiers to
	determine which process the program synchronises with by waiting for its
	termination. Forked processes execute defined programs concurrently and
	every process must only be joined once.
	Let $\Proc$ denote a finite set of programs that can be forked.
	\subsection{Syntax}
	\label{sec:syntax}
	The syntax of the examined programming language can be given by a
	context-free grammar starting in \syntax{<S>} as follows where
	$x,x_1,\dots,x_n\in\Var, s\in\Sel, t,t_1,t_2\in\PI$ and $m\in\Proc$ is
	another program:
	\setlength{\grammarindent}{2cm}
	\begin{grammar}
	<P> ::= \textbf{null} | $x$ | $x.s$

	<C> ::= <P> == <P> | <P> != <P> | <C> \&\& <C> | <C>  $\|$  <C>

	<S> ::= $x$=<P> | $x.s$ = <P> | \textbf{new}($x$) | \textbf{while}(<C>)
	\textbf{do} <S> \textbf{done}\alt
	\textbf{if} (<C>) \textbf{then} <S> \textbf{else} <S> \textbf{fi} |
	\textbf{skip} | <S>;<S>\alt
	$t$=\textbf{fork}$(m(x_1,\dots,x_n))$ | join($t$) | $t_1 = t_2$
	\end{grammar}

	\subsection{Example}
	In the following a small program is examined which traverses a binary tree.
	Therefore there is one variable called $\mathit{curr}$ which is assumed to
	initially start on a node of the binary tree. Every node has two selectors
	$\mathit{left},\mathit{right}$ identifying the left and right subtree. The
	program executes by forking another process that executes a
	different program \emph{traverse} on every node initially
	provided with the right subtree(the program traverse is assumed to not
	change the subtree it operates on) and itself moving down the left subtree.
	If there is no left subtree the most recently forked process is synchronised
	with. At last the $\mathit{left}$ selector of $\mathit{curr}$ is set to the
	right subtree of $\mathit{curr}$ and the $\mathit{right}$ selector is set to
	the designated \textbf{null} value:
	\begin{lstlisting}[caption={An example program}, label={lst:ExampleProgram}]
while($\mathit{curr.left}$ != null) do
	$\mathit{tmp}$ = $\mathit{curr.right}$;
	$t$ = fork($\mathit{traverse}$($\mathit{tmp}$));
	$\mathit{curr}$ = $\mathit{curr.left}$;
done;
$\mathit{tmp} = \mathit{curr.right}$;
$t$ = fork($\mathit{traverse}$($\mathit{tmp}$));
$\mathit{tmp} = $ null;
join($t$);
$\mathit{curr.left} = \mathit{curr.right}$;
$\mathit{curr.right} = $ null;
	\end{lstlisting}
	Thus for this program the \emph{context}(the sets $\Sel$, $\PI$, $\Var$,
	$\Proc$) is as follows:
	\begin{itemize}
		\item $\Proc = \{ \mathit{traverse} \}$
		\item $\Sel  = \{ \mathit{left}, \mathit{right} \}$
		\item $\Var  = \{ \mathit{curr}, \mathit{tmp}   \}$
		\item $\PI   = \{ t \}$
	\end{itemize}
	Furthermore consider the graphical representation of the execution of this 
	program that can be seen in Figure \ref{fig:BinTreeProgLang}.
	\begin{figure}
		\begin{minipage}{0.5\textwidth}
	\scalebox{0.64}{
	\begin{tikzpicture}[node distance=2 and 2]
		\node[node] (r)  {};
		\node[var] (curr) [above=of r] {$\mathit{curr}$};
		\node[node] (r-l) [below left=of r] {};
		\node[node] (r-r) [below right=of r] {};
		\node[node, node distance=2 and 1.2] (r-l-l) [below left=of r-l] {};
		\node[node, node distance=2 and 1.2] (r-l-r) [below right=of r-l] {};
		\node[node, node distance=2 and 1.2] (r-r-l) [below left=of r-r] {};
		\node[node, node distance=2 and 1.2] (r-r-r) [below right=of r-r] {};
		\node[node]                          (r-r-r-r) [below=of r-r-r] {};
		\node[node, node distance=2 and 0.9] (r-l-l-r) [below right=of r-l-l] {};
		\node[var] (tmp)  [above right=of r-r] {$\mathit{tmp}$};
		\node[node, below right=2 and 1 of r-l-r] (null) {$v_{\text{null}}$};
		\draw [connector] (tmp) to (r-r);
		\draw [connector] (curr) to (r);
		\draw [sel] (r) to node[right]{$\mathit{left}$} (r-l) ;
		\draw [sel] (r) to node[right]{$\mathit{right}$} (r-r);
		\draw [sel] (r-l) to node[right]{$\mathit{left}$} (r-l-l);
		\draw [sel] (r-l) to node[right]{$\mathit{right}$} (r-l-r);
		\draw [sel] (r-r) to node[right]{$\mathit{left}$} (r-r-l);
		\draw [sel] (r-r) to node[right]{$\mathit{right}$} (r-r-r);
		\draw [sel] (r-r-r) to node[right]{$\mathit{right}$} (r-r-r-r);
		\draw [sel] (r-l-l) to node[right]{$\mathit{right}$} (r-l-l-r);
		\draw [sel] (r-l-l) to node[above]{$\mathit{left}$} (null);
		\draw [sel] (r-r-r) to node[above]{$\mathit{left}$} (null);
		\draw [sel] (r-l-r) to node{$\mathit{left},\mathit{right}$} (null);
		\draw [sel] (r-r-l) to node[above]{$\mathit{left},\mathit{right}$} (null);
		\draw [sel, bend left] (r-r-r-r) to node[below]{$\mathit{left},\mathit{right}$} (null);
		\draw [sel, bend right] (r-l-l-r) to node[below]{$\mathit{left},\mathit{right}$} (null);

		\begin{scope}[on background layer]
			\node [fill=green!30,fit=(r-r) (r-r-r) (r-r-l) (r-r-r-r),label=125:$t$] {};
		\end{scope}
	\end{tikzpicture}
	}
\end{minipage}%
\begin{minipage}{0.5\textwidth}
	\scalebox{0.64}{
	\begin{tikzpicture}[node distance=2 and 2]
		\node[node] (r)  {};
		\node[node] (r-l) [below left=of r] {};
		\node[var] (curr) [above =of r-l] {$\mathit{curr}$};
		\node[node] (r-r) [below right=of r] {};
		\node[node, node distance=2 and 1.2] (r-l-l) [below left=of r-l] {};
		\node[node, node distance=2 and 1.2] (r-l-r) [below right=of r-l] {};
		\node[node, node distance=2 and 1.2] (r-r-l) [below left=of r-r] {};
		\node[node, node distance=2 and 1.2] (r-r-r) [below right=of r-r] {};
		\node[node]                          (r-r-r-r) [below=of r-r-r] {};
		\node[node, node distance=2 and 0.9] (r-l-l-r) [below right=of r-l-l] {};
		\node[var, below=of r-l-r] (tmp) {$\mathit{tmp}$};
		\draw [connector] (tmp) to (r-l-r);
		\draw [connector] (curr) to (r-l);
		\draw [sel] (r) to node[right]{$\mathit{left}$} (r-l);
		\draw [sel] (r) to node[right]{$\mathit{right}$} (r-r);
		\draw [sel] (r-l) to node[right]{$\mathit{left}$} (r-l-l);
		\draw [sel] (r-l) to node[right]{$\mathit{right}$} (r-l-r);
		\draw [sel] (r-r) to node[right]{$\mathit{left}$} (r-r-l);
		\draw [sel] (r-r) to node[right]{$\mathit{right}$} (r-r-r);
		\draw [sel] (r-r-r) to node[right]{$\mathit{right}$} (r-r-r-r);
		\draw [sel] (r-l-l) to node[right]{$\mathit{right}$} (r-l-l-r);

		\begin{scope}[on background layer]
			\node [fill=yellow!30,fit=(r-r) (r-r-r) (r-r-l) (r-r-r-r)] {};
		\end{scope}
		\begin{scope}[on background layer]
			\node [fill=green!30,fit=(r-l-r), label=45:$t$] {};
		\end{scope}
	\end{tikzpicture}
	}
\end{minipage}
\\[1cm]
\begin{minipage}{0.5\textwidth}
	\scalebox{0.6}{
		\begin{tikzpicture}[node distance=2 and 2]
			\node[node] (r)  {};
			\node[node] (r-l) [below left=of r] {};
			\node[node] (r-r) [below right=of r] {};
			\node[node, node distance=2 and 1.2] (r-l-l) [below left=of r-l] {};
			\node[var] (curr) [above =of r-l-l] {$\mathit{curr}$};
			\node[node, node distance=2 and 1.2] (r-l-r) [below right=of r-l] {};
			\node[node, node distance=2 and 1.2] (r-r-l) [below left=of r-r] {};
			\node[node, node distance=2 and 1.2] (r-r-r) [below right=of r-r] {};
			\node[node]                          (r-r-r-r) [below=of r-r-r] {};
			\node[node, node distance=2 and 0.9] (r-l-l-r) [below right=of r-l-l] {};
			\node[var, left=of r-l-l-r] (tmp) {$\mathit{tmp}$};
			\draw [connector] (tmp) to (r-l-l-r);
			\draw [connector] (curr) to (r-l-l);
			\draw [sel] (r) to node[right]{$\mathit{left}$} (r-l);
			\draw [sel] (r) to node[right]{$\mathit{right}$} (r-r);
			\draw [sel] (r-l) to node[right]{$\mathit{left}$} (r-l-l);
			\draw [sel] (r-l) to node[right]{$\mathit{right}$} (r-l-r);
			\draw [sel] (r-r) to node[right]{$\mathit{left}$} (r-r-l);
			\draw [sel] (r-r) to node[right]{$\mathit{right}$} (r-r-r);
			\draw [sel] (r-r-r) to node[right]{$\mathit{right}$} (r-r-r-r);
			\draw [sel] (r-l-l) to node[right]{$\mathit{right}$} (r-l-l-r);

			\begin{scope}[on background layer]
				\node [fill=yellow!30,fit=(r-r) (r-r-r) (r-r-l) (r-r-r-r)] {};
			\end{scope}
			\begin{scope}[on background layer]
				\node [fill=yellow!30,fit=(r-l-r)] {};
			\end{scope}
			\begin{scope}[on background layer]
				\node [fill=green!30,fit=(r-l-l-r), label=45:$t$] {};
			\end{scope}
		\end{tikzpicture}
	}
\end{minipage}%
\begin{minipage}{0.5\textwidth}
	\scalebox{0.64}{
		\begin{tikzpicture}[node distance=2 and 2]
			\node[node] (r)  {};
			\node[node] (r-l) [below left=of r] {};
			\node[node] (r-r) [below right=of r] {};
			\node[node, node distance=2 and 1.2] (r-l-l) [below left=of r-l] {};
			\node[var] (curr) [above =of r-l-l] {$\mathit{curr}$};
			\node[node, node distance=2 and 1.2] (r-l-r) [below right=of r-l] {};
			\node[node, node distance=2 and 1.2] (r-r-l) [below left=of r-r] {};
			\node[node, node distance=2 and 1.2] (r-r-r) [below right=of r-r] {};
			\node[node]                          (r-r-r-r) [below=of r-r-r] {};
			\node[node, node distance=2 and 0.9] (r-l-l-r) [below left=of r-l-l] {};
			\draw [connector] (curr) to (r-l-l);
			\draw [sel] (r) to node[right]{$\mathit{left}$} (r-l);
			\draw [sel] (r) to node[right]{$\mathit{right}$} (r-r);
			\draw [sel] (r-l) to node[right]{$\mathit{left}$} (r-l-l);
			\draw [sel] (r-l) to node[right]{$\mathit{right}$} (r-l-r);
			\draw [sel] (r-r) to node[right]{$\mathit{left}$} (r-r-l);
			\draw [sel] (r-r) to node[right]{$\mathit{right}$} (r-r-r);
			\draw [sel] (r-r-r) to node[left]{$\mathit{right}$} (r-r-r-r);
			\draw [sel] (r-l-l) to node[right]{$\mathit{left}$} (r-l-l-r);

			\begin{scope}[on background layer]
				\node [fill=yellow!30,fit=(r-r) (r-r-r) (r-r-l) (r-r-r-r)] {};
			\end{scope}
			\begin{scope}[on background layer]
				\node [fill=yellow!30,fit=(r-l-r)] {};
			\end{scope}
		\end{tikzpicture}
	}
\end{minipage}


		\caption{A graphical representation of the execution of the example
		program for one possible heap state, note that aside from the first step
		selectors which refer to the \textbf{null} value are omitted.}
		\label{fig:BinTreeProgLang}
	\end{figure}
	Some interesting observations can be made:
	\begin{enumerate}
		\item Because of the structure of the binary tree the newly forked
			process can only access the subtrees starting from the parameter
			object (marked in green).
		\item After starting a new process the previously started process cannot
			be referred to anymore since the process identifier is assigned to
			another process. Thus there is no information about any process
			operating on the yellow marked areas of the binary tree.
		\item The last process is joined again, thus it can be assured that it
			has terminated after the join statement. This means especially that
			no process operates on the subtree which is moved from the right hand
			side to the left hand side (as indicated by the absence of a yellow
			mark).
	\end{enumerate}
	The in the following presented analysis of programs relies heavily on
	representing the current state as graph. But in order to approach parallel
	execution of multiple processes which share heap objects and their selectors
	this representation is furtherly enriched. In order to account which parts
	of the heap are accessible by multiple processes it is locally logged for
	the selectors to which processes these are visible.
